
\section{Prepare data}
\label{sec:dataPrepare}
\subsection{Raw data}

Data should be prepared as the example in Figure~\ref{fig:dataMicroarray}.
First column should be feature ID (e.g. gene symbol) and the rest of the columns are samples.
Note that the first column can also be other feature type (i.e. probe id, entrez ID).
The first row is sample ID.
Valid data type includes continuous data and count data.

\begin{figure}[H]
\begin{center}
\includegraphics[scale=0.5]{./figure/dataPreparation/dataMicroarray}
\caption{A example input data format}
\label{fig:dataMicroarray}
\end{center}
\end{figure}

\subsection{Clinical data}

Clinical data should be prepared as the example in Figure~\ref{fig:clinical}.
First column should be sample ID and each row represents a sample.
The rest of the columns are clinical information.

\begin{figure}[H]
\begin{center}
\includegraphics[scale=0.5]{./figure/dataPreparation/clinicalData}
\caption{A example clinical data format}
\label{fig:clinical}
\end{center}
\end{figure}

\subsection{Example data with the MetaOmics software}

\subsubsection{Leukemia datasets}

We collected three studies from NCBI GEO website.   
The original datasets are due by \cite{verhaak2009prediction}, \cite{balgobind2011evaluation} and \cite{kohlmann2008international}.	
This this example we  considered samples from acute myeloid leukemia (AML) with subtype 
	inv(16)(inversions in chromosome 16), 
	t(15;17)(translocations between chromosome 15 and 17), 
	t(8;21)(translocations between chromosome 8 and 21).
	These AML subtypes have been well-studied with different survival, 
	treatment response and prognosis outcomes.
	

			\begin{table}	
			\caption{Multi-study leukemia gene expression profiles. All three studies are from Affymetrix Human Genome U133plus2 with 5,135 genes.
		Three subtypes of leukemia are defined as the chromosomal translocation,
		including inversion of chromosome 16 - inv(16), translocation of chromosome 15 and 17 - t(15:17) and 
		translocation of chromosome 8 and 21 - t(8:21).}						
			\centering
\begin{tabular}{c  c  c   c   }
  \hline 
  \hline 
\multirow{2}*{Study}   & \multirow{2}*{source}   & \multirow{2}*{\# samples}  & \# samples by subtypes \\
 & & & inv(16)/t(15:17)/t(8,21)  \\
  \hline 
Study 1 & \cite{verhaak2009prediction} & 89 & 33/21/35\\
Study 2 & \cite{balgobind2011evaluation} & 74 & 27/19/28\\
Study 3 & \cite{kohlmann2008international} & 105 & 28/37/40\\
  \hline 
  \hline 
\end{tabular}
			\label{tab:realDataLeukemia}
		\end{table}
		
			\begin{table}	
			\caption{Multi-study breast cancer gene expression profiles. 
			Each gene expression profiles of all four studies contain 10,330 genes.
			Study 1 contains both count data and fpkm (continuous) data so user should {\bf select only one of them}. 
			The other three studies contain only continuous data.
			The phenotype of interest is estrogen-receptor (comparing ER+ vs ER-).}						
			\centering
			\begin{tabular}{c  c  c   c  c  }
			  \hline 
			  \hline 
			\multirow{2}*{Study}   & \multirow{2}*{source}   & \multirow{2}*{scale}  & \multirow{2}*{\# samples}  & \# samples by ER \\
 & & & & ER+/ER-  \\
  \hline 
\multirow{2}*{Study 1}  & \multirow{2}*{\cite{weinstein2013cancer}}  & count & \multirow{2}*{406} & \multirow{2}*{319/87}\\
&& continuous && \\
Study 2 & \cite{desmedt2007strong} & continuous &  198 & 134/64\\
Study 3 & \cite{wang2005gene} & continuous & 286 & 209/77\\
Study 4 & \cite{ivshina2006genetic} & continuous & 245 & 211/34\\
  \hline 
  \hline 
\end{tabular}
			\label{tab:realDataLeukemia1}
		\end{table}

\subsubsection{Prostate cancer datasets}

			\begin{table}	
			\caption{Prostate cancer dataset information. Eight prostate cancer gene expression profiles were measured by different microarray platforms.}						
			\centering
	\begin{tabular}{c c c c c}
	\hline
	\hline
\multirow{2}*{Study}   & \multirow{2}*{source}   & \multirow{2}*{\# samples}  & \# samples by label  & \multirow{2}*{\# genes}\\
& & & Normal/Primary& \\
	\hline
	Study 1 & \cite{welsh2001analysis} &  34 & 9/25 & 8798 \\
	Study 2 & \cite{yu2004gene} &  146 & 81/65 & 8799 \\
	Study 3 & \cite{lapointe2004gene} &  103 & 41/62 & 13579 \\
	Study 4 & \cite{varambally2005integrative} &  13 & 6/7 & 19738 \\
	Study 5 & \cite{singh2002gene}  &  102 & 50/52  & 8799 \\
	Study 6 & \cite{wallace2008tumor} &  89 & 20/69 & 12689  \\
	Study 7 & \cite{nanni2006epithelial} &  30 & 7/23  & 12689 \\
	Study 8 & \cite{tomlins2006tmprss2} &  57 & 27/30 & 9703   \\
	\hline
	\hline
	\label{tab:prostate}
	\end{tabular}
			\label{tab:realDataLeukemia2}
		\end{table}

\subsubsection{Prostate cancer datasets}



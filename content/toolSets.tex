\section{Toolsets}

After running the MetaPreprocess module,
all the preparatory steps are done. 
It's time to apply the seven analytical modules in MetaOmics by clicking on the ``Toolsets" tab and select the tool for your research question.
In the next few subsections, 
we will introduce in details how to run each of these modules.
For each module, a summary table of studies and sample sizes is shown on the top left corner. 
There is an ``about" drop-down menu which contains a brief introduction and tutorial of the module.
The ``options" drop-down menu contain common options users can select or tune in the analysis.
The ``advanced options" menu contains more technical options which we generally do not recommend users to change unless they are familiar with the methods.
After applying the modules, all result files will be automatically saved in the working directory which is specified in Section~\ref{sec:setting}.
For computationally demanding methods, the procedure may take minutes or up to hours depending on size of datasets.
Users can keep track of the progress by checking the R console.








After starting metaOmics, 
the first page is the metaOmics setting page in Figure~\ref{fig:GUIsetting}.  
There are 4 tabs on top of the page (at position {\color{red} (1)}): Setting, Preprocessing, Saved Data and Toolsets.
Below the 4 tabs, 
the first header is the session information.
{
\color{blue}
Why do we need session information?
}
The second header is Directory for Saving Output Files (at position {\color{red} (2)}).
By clicking $\ldots$,
user can set default working directory, in which all the meta-analysis results will be saved.
User can view their current working directory on the top right corner (at position {\color{red} (3)}).
The third header is Toolsets (at position {\color{red} (4)}),
here users can view if individual packages are installed.
If the packages are installed, there is a checked installed status.
Otherwise, users can install individual package by clicking install blue button.
Position {\color{red} (5)} shows the current active dataset, which will be introduced in Section~\ref{sec:procedure}~\ref{sec:active}
 
\begin{figure}[H]
\begin{center}
\includegraphics[scale=0.35]{./figure/preprocessing/GUIsetting}
\caption{GUI setting page}
\label{fig:GUIsetting}
\end{center}
\end{figure}

\subsection{Preprocessing}

In this subsection, we will introduce how to upload your dataset into the MetaOmics suit such that the functional modules can process the uploaded datasets.


\subsubsection{Procedure}
\label{sec:procedure}

\begin{steps}
\item \textbf{Uploading data:}

\begin{figure}[!htbp]
\begin{center}
\includegraphics[scale=0.35]{./figure/preprocessing/GUIpreprocessing}
\caption{GUI Preprocessing page}
\label{fig:GUIpreprocessing}
\end{center}
\end{figure}

If users go to the Preprocessing tab as in Figure~\ref{fig:GUIpreprocessing},
they are able to upload genomic data via the tab ``Choosing/Upload Expression Data" as in Figure~\ref{fig:GUIpreview} (at position {\color{red} (1)}).
The data should be prepared according to Section~\ref{sec:dataPrepare}.
Users may optionally upload Clinical Data (at position {\color{red} (2)}), depending on biological purpose.
The all MetaOmics modules except for MetaClust  require external clinical labels.
The MetaOmics suit also provides handlers (at position {\color{red} (3)}) for feature annotation, missing value imputation and multiple probe same genes.
After uploading is complete,
users can preview their data on the right hand side of the page as Figure~\ref{fig:GUIpreview}.

\item \textbf{Preprocessing:}


There are several expression data parsing option available on the left panel of Figure~\ref{fig:GUIpreview}.
A complete introduction of these options are available at the end of this subsection.
The right hand side of Figure~\ref{fig:GUIpreview} shows the summary statistics of uploaded data and preview of the data matrix.
There is a search box such that the user can search their favorite genes.

\begin{figure}[H]
\begin{center}
\includegraphics[scale=0.35]{./figure/preprocessing/GUIpreview}
\caption{GUI Preprocessing page}
\label{fig:GUIpreview}
\end{center}
\end{figure}
After users upload clinical data (e.g. case control labels) and specify type of data and study name.
They can click ``save single study" button, single study will be saved.

\item \textbf{Saved Data:}

After uploading multiple studies w/o clinical data,
Users can turn to the Saved Data tab.
Users should select multiple datasets as Figure~\ref{fig:GUImerge} (at position {\color{red} (2)}).
\begin{figure}[H]
\begin{center}
\includegraphics[scale=0.35]{./figure/preprocessing/GUImerge}
\caption{GUI Preprocessing page}
\label{fig:GUImerge}
\end{center}
\end{figure}
Users can select filtering criteria, enter merged study name and click on the Merge from Selected Datasets (at position {\color{red} (1)}).
A merged dataset will appear on the  ``List of saved data" panel (at position {\color{red} (2)}).

\item \textbf{Make merged Dataset Active:}

\label{sec:active}
The last thing users need to do before using meta-analytic toolsets is select merged data and click on 
``Make merged Active Dataset" - A big green button (at position {\color{red} (1)}).
Then the merged data becomes active study and shows up on the top right corner (at position {\color{red} (4)}).
The active dataset serves as the input for all other MetaOmics modules.
If users want to delete a dataset, just click ``Delete Selected Data" button (at position {\color{red} (3)}) after selection the dataset.

\end{steps}

\textbf{Complete List of Options:} 
\begin{enumerate}
\item Upload expression data:
\begin{itemize}
\item Header: should be checked if the input file includes a header.
\item Separator: indicates what type of separator is used for the data matrix.
\item Quote for String: how is the data matrix quoted.
\item Log transforming data: if you want to perform log transformation of your data, check yes.
\item Use existing datasets: if you want to load a dataset previously uploaded, you can choose from the checklist.
\end{itemize}
\item Annotation/impute/Replicate:
\begin{itemize}
\item Annotation: possible ID type can be Gene Symbol (default), Probe ID, reference sequence ID, entrez ID.
\item Impute: if selected, missing value imputation will be performed by k-nearest neighbor algorithm.
\item Replicate Handling: if selected, if the same gene symbol maps to multiple probes, the probe with the largest inner quantile range (IQR) will be selected.
\end{itemize}
\item Saved Data, Merging and Filtering Datasets:
\begin{itemize}
\item Mean: the criteria such that how many percent of genes will be filtered out based on sum of mean ranks (e.g. 0.3 represent 30\%).
\item Variance: after the Mean filtering, the criteria such that how many percent of genes will be filtered out based on sum of variance ranks (e.g. 0.3 represent 30\%).
\item Study Name: dataset name after merging. This name will appear in the list of saved data table.
\item Merge from Selected Datasets: perform filtering and merging.
\end{itemize}
\item Danger zone:
\begin{itemize}
\item Delete Selected Data: the selected data will be delete permanently if clicked, so please be cautious.
\end{itemize}

\end{enumerate}




\section{Complete list of options}
\label{sec:completeList}

\subsection*{MetaPreprocess}
\label{sec:completeList_MetaPreprocess}


\textbf{Complete List of Options:} 
\begin{enumerate}
\item Upload expression data:
\begin{itemize}
\item Header: should be checked if the input file includes a header.
\item Separator: indicates what type of separator is used for the data matrix.
\item Quote for String: how is the data matrix quoted.
\item Log transforming data: if you want to perform log transformation of your data, check yes.
\item Use existing datasets: if you want to load a dataset previously uploaded, you can choose from the checklist.
\end{itemize}
\item Annotation/impute/Replicate:
\begin{itemize}
\item Annotation: possible ID type can be Gene Symbol (default), Probe ID, reference sequence ID, entrez ID.
\item Impute: if selected, missing value imputation will be performed by k-nearest neighbor (KNN) algorithm.
\item Replicate Handling: if selected and if the same gene symbol maps to multiple probes, the probe with the largest interquartile range (IQR) will be selected
as a representative for this gene.
\end{itemize}
\item Saved Data, Merging and Filtering Datasets:
\begin{itemize}
\item Mean: the percentage of genes being filtered out based on the mean expression levels (e.g. 0.3 represent 30\%).
\item Variance: the percentage of genes being filtered out based on the variance of expression (e.g. 0.3 represent 30\%).
\item Study Name: dataset name after merging. This name will appear in the list of saved data table.
\item Merge from Selected Datasets: perform filtering and merging.
\end{itemize}
\item Danger zone:
\begin{itemize}
\item Delete Selected Data: the selected data will be deleted permanently if clicked, so please be cautious.
\end{itemize}

\end{enumerate}

\subsection{MetaQC}
\label{sec:completeList_MetaQC}

\textbf{Complete List of Options:} 

\begin{enumerate}
  \item Options
  \begin{itemize}
     \item Perform gene filtering: If yes: cut lowest percentile by mean, cut lowest percentile by variance. 
     \item Use adjusted p-value for selecting DE genes
     \item p-value cutoff for selecting DE genes
     \item Use adjusted p-value for selecting pathways
     \item p-value cutoff for selecting pathways
    \end{itemize}
   \item Advanced Option (**Optional): 
        \begin{itemize}
      \item Pathway min gene size
      \item Pathway max gene size
      \item Number of permutations
    \end{itemize} 
  \item Run MetaQC Analysis
\end{enumerate}
   
\subsection{MetaDE}
\label{sec:completeList_MetaDE}

\textbf{Complete List of Options:} 

\begin{enumerate}
  \item Meta Method Type: Combining p-value, Combining effect size, Others.
  \item Meta Method: Fisher, AW-Fisher, FEM, REM, Sum of Rank, Product of Rank, multi-class correlation 
  \item Mixed data type: selected if both count data and continuous data exist.
  \item Response Type:
   \begin{itemize}
     \item Two class comparison, Multi-class comparison, Continuous outcome, Survival outcome.
     \item Label Attribute: select the label name of the outcome.
     \item Control Label \& Experimental Label: specify the case/control label for two-class comparison.
    \end{itemize}
   \item Individual Study Option:
     \begin{itemize}
     \item Setting individual study method
     \item Setting individual study paired option
    \end{itemize} 
   \item Advanced Option (**Optional):
     \begin{itemize}
      \item Use complete options
      \item Parametric
      \item Covariate
      \item Alternative hypothesis
    \end{itemize} 
    \item Run
    \item Pathway Databases
    \item Pathway Analysis Option:
         \begin{itemize}
      \item Pathway enrichment method
      \item Pathway min gene size
      \item Pathway max gene size
    \end{itemize} 
    \item Run Pathway Analysis
\end{enumerate}



\subsection{MetaPath}
\label{sec:completeList_MetaPath}


\textbf{Complete List of Options:} 

\begin{enumerate}
  \item mixed data types: whether the input data is a mixture of count data and continuous data.
  \item Response Type:
   \begin{itemize}
     \item Two class comparison, Multi-class comparison, Continuous outcome, Survival outcome.
     \item Label Attribute: select the label name of the outcome.
     \item Control Label \& Experimental Label: specify the case/control label for two-class comparison.
    \end{itemize}
   \item Individual Study Option:
     \begin{itemize}
     \item Setting individual study method
     \item Setting individual study paired option
    \end{itemize} 
   \item Advanced Option (**Optional):
     \begin{itemize}
      \item Covariate
      \item Alternative hypothesis
    \end{itemize} 
    \item Pathway Databases
    \item Pathway Analysis Option:
         \begin{itemize}
       \item Software
      \item Pathway enrichment method
      \item Pathway min gene size
      \item Pathway max gene size
    \end{itemize} 
    \item Run Pathway Analysis
    \item Pathway Clustering Diagnostics
    \item Get Clustering Result
\end{enumerate}


\subsection{MetaNetwork}
\label{sec:completeList_MetaNetwork}

\textbf{Complete List of Options:} 
\begin{enumerate}
\item Generate Network:
\begin{itemize}
\item Case Name: specify case group label.
\item Control Name: specify control group label.
\item Number of Permutations: the number of permutations used for generating network.
\item Edge Cutoff: edge cut-off  determines the proportion of edges to be kept in the network.
\end{itemize}

\item Search for basic modules:
\begin{itemize}
\item Number to repeat:  the number of repeats used for each initial seed modules.
\item MC steps:  the maximum Monte Carlo steps for simulated annealing algorithm.
\item Jaccard cutoff: maximum pairwise Jaccard index allowed for basic modules.
If two repeats from Monte Carlo simulation are very similar (with Jaccard index greater than the cutoff),
only one repeat with stable configuration (low energy) will be kept in the analysis.
\end{itemize}
\item Assemble supermodules:

\begin{itemize}
\item FDR cutoff:  FDR cut-off to select basic modules for supermodule assembly.
\end{itemize}
\end{enumerate}

\textbf{Intermediate results}
 
\begin{enumerate}
\item Generate Network
\begin{itemize}
\item AdjacencyMatrices.Rdata is a list of adjacency matrices for case and control subjects in each study. The order is study1 case, study2 case, \dots, studyS case, study1 control, study2 control, \dots, studyS control.
\item CorrelationMatrices.Rdata is a list of correlation matrices for case and control subjects in each study.
\item AdjacencyMatricesPermutationP.Rdata is a list of adjacency matrices for permuted datasets in permutation P.
\end{itemize}

\item Search for basic modules

\begin{itemize}
 \item basic\_modules\_summary\_forward\_weight\_w1.csv is a summary table of basic modules that are higher correlated in case, detected using w1.
 \item basic\_modules\_summary\_backward\_weight\_w1.csv is a summary table of basic modules that are higher correlated in control, detected using w1.
\item threshold\_forward.csv is a table of number of basic modules higher correlated in case, detected under different w1 values and FDR cut-offs.
\item threshold\_backward.csv is a table of number of basic modules higher correlated in control, detected under different w1 values and FDR cut-offs.
 \item permutation\_energy\_forward\_P.Rdata is a list of energies for basic modules that higher correlated in case, detected from permutation P.
  \item permutation\_energy\_backward\_P.Rdata is a list of energies for basic modules that higher correlated in control, detected from permutation P.
\end{itemize}

\item Assemble supermodules
 \begin{itemize}
\item module\_assembly\_summary\_weight\_w1.csv is summary table of supermodules using w1 weight.
\item CytoscapeFiles folder contains the input files for Cytoscape to visualize supermodules.
\end{itemize}

\end{enumerate}
 

\subsection{MetaPredict}
\label{sec:completeList_MetaPredict}



\textbf{Complete List of Options:} 
\begin{enumerate}
\item Model trainings: 
\begin{itemize}
\item Methods for MetaPredict: include Mean score, Fisher, Stouffer.
\item Max number of top scoring pairs (K)
\item Number of cores for parallel computing
\item TWO labels to cluster: labels for MetaPredict
\item Please select studies for training
\item Please select studies for testing
\item Number of top scoring pairs (K): Number of top scoring pairs (K) for prediction.
\end{itemize}

\end{enumerate}


\subsection{MetaClust}
\label{sec:completeList_MetaClust}


\textbf{Complete List of Options:} 
\begin{enumerate}
\item Tune $K$ (** optional)
\begin{itemize}
\item Maximum of $K$: the maximum number of $K$ that gap statistics will step through.
\item Top percentage by larger variance: Top percentage p\% by larger variance means that we will use top p\% larger variance genes to perform gap statistics.
\item Number of permutaitons: Number of permutation is number of bootstrap samples for gap statistics.
\item Select studies to be tuned: Studies to be tuned.
\item Tune $K$: start tuning $K$.
\end{itemize}
\item Tune Wbounds (** optional)
\begin{itemize}
\item Number of clusters for tuning wbounds: number of clusters for tuning Wbounds.
\item Iterations: Iterations are number of bootstrap samples for gap statistics.
\item Minimum of wbounds: lower bound of the searching space of Wbounds.
\item Maximum of wbounds: upper bound of the searching space of Wbounds.
\item Step of of wbounds: stepsize of the searching space of Wbounds.
\item Tune wbounds: start tuning wbounds.
\end{itemize}
\item Run Meta Sparse $K$-means: 
\begin{itemize}
\item Number of clusters: number of clusters. Can be tuned from Tune $K$ option.
\item Wbounds: control numbers of selected features. Can be tuned from Tune Wbounds option.
\item Methods for MetaClust: Exhaustive is suggested if the data is not large.
Linear will perform smart search and get solution much faster than Exhaustive, 
but it may yield less accuracy.
MCMC might by very time consuming.
\item Adjust sample size: adjust sample size effect.
\item Run meta sparse Kmeans: start tuning wbounds.
\end{itemize}

\end{enumerate}

\subsection{MetaPCA}
\label{sec:completeList_MetaPCA}

\textbf{Complete List of Options:} 
\begin{enumerate}
\item Common MetaPCA parameters: 

\begin{itemize}
\item Methods for MetaPCA:
SSC represent MetaPCA via sum of squared cosine (SSC) maximization.
SV represent MetaPCA via sum of variance decomposition (SV).
\item Dimension of meta-eigenvector matrix: dimension of the output meta-eigenvector matrix.
\item Dimension determined by variance quantile:
the dimension size of each study's eigenvector matrix (SSC) is determined  by the pre-defined level of variance quantile 80\%.
\end{itemize}

\item If sparsity encouraged is selected, there are extra tuning parameter ($\lambda$) that may need to be tuned.

\begin{itemize}
\item  Min $\lambda$: lower bound of the searching space of $\lambda$.
\item Max $\lambda$: upper bound of the searching space of $\lambda$.
\item Step of of $\lambda$: stepsize of the searching space of $\lambda$.
\item Tuning parameter for sparsity: Tuning parameter for sparsity that will be used for sparse MetaPCA.
\end{itemize}


\end{enumerate}


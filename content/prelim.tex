\section{Preliminaries}
\subsection{Citing MetaOmics}
MetaOmics implements many meta-analytic methodology by their authors. 
Please cite appropriate papers when you use result from MeteOmics suit,
by which the authors will receive professional credit for their work.

\begin{itemize}
\item MetaOmics suit itself can be cited as:
\item MetaQC: \bibentry{kang2012metaqc}.
\item MetaDE: 
\begin{itemize}
\item \bibentry{fisher1925statistical}.
\item \bibentry{li2011adaptively}.
\item \bibentry{choi2003combining}.
\item and many more
\end{itemize}
\item MetaPath: 
\begin{itemize}
\item \bibentry{shen2010meta}.
\item \bibentry{fang2016cpi}.
\end{itemize}
\item MetaClust: \bibentry{huo2016meta}.
\item MetaPCA: not published yet.
\item MetaKTSP: not published yet.
\item MetaDCN: not published yet.
\item MetaLA: not published yet.
\end{itemize}


\subsection{Installation}
The full instruction of how to install, start are available at \url{https://github.com/metaOmic/metaOmics}.
\subsubsection{Requirement}
\begin{itemize}
\item R $>=$ 3.3.1
\item Shiny $>=$ 0.13.2
\end{itemize}

\subsubsection{How to start the app}
\begin{itemize}
\item First, clone the project
\item git clone https://github.com/metaOmic/metaOmics
\item in R (suppose the application directory is metaOmics),

$>$ install.packages($'$shiny$'$)

$>$ shiny::runApp(`metaOmics', port=9987, launch.browser=T)
\end{itemize}

\subsubsection{How to start the documentation}
\begin{itemize}
\item Install rmarkdown for R
\item Inside `doc' directory, start R console, and:
\item in R
\begin{lstlisting}[language=R]
rmarkdown::run(shiny_args=list(port=9988, launch.browser=T))
\end{lstlisting}

\item or in command line
\begin{lstlisting}[language=bash]
R -e "rmarkdown::run(shiny_args=list(port=9988, launch.browser=T))"
\end{lstlisting}

If you run into an issue with something like `pandoc version 1.12.3 or higher is required and was not found.?, just install pandoc manually. For example, on Mac, it would be `brew install pandoc'. If you have Rstudio, you can also to get rstudio's pandoc environment. Go to rstudio console and find the system environment variable for `RSTUDIO\_PANDOC'

In R: 

\begin{lstlisting}[language=R]
Sys.getenv("RSTUDIO_PANDOC")
\end{lstlisting}

\end{itemize}


\subsection{Question and bug report}
{
\color{red}
Ask Anzhe what is the appropriate way to maintain the package?
}

 

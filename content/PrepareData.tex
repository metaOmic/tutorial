
\section{Prepare data}
\label{sec:dataPrepare}
\subsection{Raw data}

Data should be prepared as the example in Figure~\ref{fig:dataMicroarray}.
First column should be feature ID (e.g. gene symbol) and the rest of the columns are samples.
Note that the first column can also be other feature type (i.e. probe id, entrez ID).
The first row is sample ID.
Valid data type includes continuous, count.

\begin{figure}[H]
\begin{center}
\includegraphics[scale=0.5]{./figure/dataPreparation/dataMicroarray}
\caption{A example data format}
\label{fig:dataMicroarray}
\end{center}
\end{figure}

\subsection{Clinical data}

Clinical data should be prepared as the example in Figure~\ref{fig:clinical}.
First column should be sample ID and each row represents a sample.
The rest of the columns are clinical information.

\begin{figure}[H]
\begin{center}
\includegraphics[scale=0.5]{./figure/dataPreparation/clinicalData}
\caption{A example clinical data format}
\label{fig:clinical}
\end{center}
\end{figure}

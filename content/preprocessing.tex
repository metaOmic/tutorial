

\section{MetaPreprocess}

In this section, we introduce how to upload the datasets into the MetaOmics software suite to run any of the analytical modules.
The R package for MetaPreprocess module can be found at \url{https://github.com/metaOmics/preproc}.

\subsection{Preprocessing page}
\label{sec:procedure}

\begin{steps}
\item \textbf{Uploading data:}

\begin{figure}[!htbp]
\begin{center}
\includegraphics[scale=0.7]{./figure/preprocessing/GUIpreprocessing}
\caption{GUI Preprocessing page}
\label{fig:GUIpreprocessing}
\end{center}
\end{figure}

In the Preprocessing page,
users can click on ``Choosing/Upload Expression Data" tab to upload individual expression data file (i.e. data file from each study) or choose the existing saved data file as in Figure~\ref{fig:GUIpreprocessing} {\color{red} (1)}.
The data should be prepared according to Section~\ref{sec:dataPrepare}.
Users may optionally upload Clinical Data {\color{red} (2)}, depending on their biological purposes.
All MetaOmics modules except for MetaClust require external clinical labels.
The three example datasets are available under MetaOmics folder ``metaOmics/data/example/",
and we will mainly focus on the leukemia dataset (``metaOmics/data/example/") throughout this tutorial.

\item \textbf{Preprocessing:}

The MetaOmics software suite also provides handlers {\color{red} (3)} for feature annotation, missing value imputation and multiple probe same genes.
After the csv file for gene expression profile is specified, 
users can preview their data on the right side of the page.
Several expression data parsing options (e.g. header, column separator, etc) are available on the left panel of Figure~\ref{fig:GUIpreview}.
For preprocessing, 
click on ``Annotate/Impute/Replicate" to 
\begin{enumerate}
\item annotate the probe ID/reference sequence ID/Entrez ID of individual dataset (choose Gene Symbol if the input data rows are already annotated).
\item impute missing value using k-nearest neighbors (KNN) algorithm.
\item handle the multiple probes matching to the same gene issue.
\end{enumerate}

A complete introduction of these options is available at the end of this subsection.
The right side of Figure~\ref{fig:GUIpreview} will show the summary statistics of uploaded data and preview of the data matrix.
There is a search box where users can search for their genes of interest.

\begin{figure}[H]
\begin{center}
\includegraphics[scale=0.7]{./figure/preprocessing/GUIpreview}
\caption{Uploading individual studies}
\label{fig:GUIpreview}
\end{center}
\end{figure}

\item \textbf{Save single study:}
In the next step,
specify the data type (microarray data; RNA-seq FPKM/RPKM data; RNA-seq count data) and study name,
click ``save single study".
To upload RNA-seq data, the count data and FPKM/TPM
 data should be uploaded separately and saved using different names.

\item \textbf{Upload datasets for all studies:}
Repeat the steps above for all studies in the meta-analysis.
All uploaded studies are now available in the ``Saved Data" page. 
 
\end{steps}

\subsection{Saved Data page}
\label{sec:saved}

After uploading multiple studies with or without clinical data,
users can turn to the Saved Data page.

\begin{figure}[H]
\begin{center}
\includegraphics[scale=0.9]{./figure/preprocessing/GUImerge.pdf}
\caption{Merge from selected datasets}
\label{fig:GUImerge}
\end{center}
\end{figure}


\begin{steps}
\item \textbf{Merging and Filtering:}
All saved datasets from the previous step will be found in  Figure~\ref{fig:GUImerge} {\color{red} (2)}.
Users should select multiple datasets for further meta-analysis purpose.
Users can filter out genes with low expression level (by default, mean expression lower than $30^{th}$ percentile)
or low variance (by default, variance lower than $30^{th}$ percentile).
% a project with stringent filtering criteria keeps  500-1000 genes after the filtering step.
After specifying filtering criteria, enter Project Name and click on the ``Merge from Selected Datasets" {\color{red} (1)}.
A merged dataset (study type = ``multiple") will appear on the ``List of saved data" panel {\color{red} (3)}.
Creating multiple projects with varying preprocessing criteria is useful.
For example, the user can start from a project with harsh filtering criteria (maintain 500-1000 genes) and give a test run through all modules to save time.
If successful, a larger project can be created and implemented.
If users want to delete any dataset, they can click the red danger zone button and delete the selected datasets.

\item \textbf{Make active dataset:}
\label{sec:active}
The last thing to do before using any of the meta-analytic modules is to select the merged data and click on 
``Make your dataset Active Dataset" - A big green button in Figure~\ref{fig:active}.
Then the merged data will become the active study that shows up on the top right corner of the page.
The active dataset serves as the input for all the analytical modules in MetaOmics.

\end{steps}







\begin{figure}[H]
\begin{center}
\includegraphics[scale=0.7]{./figure/preprocessing/GUImarkActive}
\caption{Make merged Dataset Active}
\label{fig:active}
\end{center}
\end{figure}




\textbf{Complete List of Options:} 
\begin{enumerate}
\item Upload expression data:
\begin{itemize}
\item Header: should be checked if the input file includes a header.
\item Separator: indicates what type of separator is used for the data matrix.
\item Quote for String: how is the data matrix quoted.
\item Log transforming data: if you want to perform log transformation of your data, check yes.
\item Use existing datasets: if you want to load a dataset previously uploaded, you can choose from the checklist.
\end{itemize}
\item Annotation/impute/Replicate:
\begin{itemize}
\item Annotation: possible ID type can be Gene Symbol (default), Probe ID, reference sequence ID, entrez ID.
\item Impute: if selected, missing value imputation will be performed by k-nearest neighbor (KNN) algorithm.
\item Replicate Handling: if selected and if the same gene symbol maps to multiple probes, the probe with the largest interquartile range (IQR) will be selected
as a representative for this gene.
\end{itemize}
\item Saved Data, Merging and Filtering Datasets:
\begin{itemize}
\item Mean: the percentage of genes being filtered out based on the mean expression levels (e.g. 0.3 represent 30\%).
\item Variance: the percentage of genes being filtered out based on the variance of expression (e.g. 0.3 represent 30\%).
\item Study Name: dataset name after merging. This name will appear in the list of saved data table.
\item Merge from Selected Datasets: perform filtering and merging.
\end{itemize}
\item Danger zone:
\begin{itemize}
\item Delete Selected Data: the selected data will be deleted permanently if clicked, so please be cautious.
\end{itemize}

\end{enumerate}




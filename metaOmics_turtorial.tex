\documentclass{article}

\usepackage[utf8]{inputenc}
\usepackage{hyperref}
\usepackage{natbib}
\usepackage{bibentry}
\usepackage{color}
\usepackage{graphicx}
\usepackage{float}

\nobibliography*

\title{A tutorial for metaOmic}
\author{}
\date{ }
 
 
\begin{document}
 
\maketitle
 
\tableofcontents
 
\section{Introduction}
 
MetaOmics software suite is an interactive software with graphical user interface (GUI) for genomic meta-analysis implemented using R shiny.
Many state of art meta-analysis tools are available in this software,
including MetaProcess for omics data preprocessing, 
MetaQC for quality control, 
MetaDE for differential expression analysis,
MetaPath for pathway enrichment analysis,
MetaNetwork for differential co-expression network analysis,
MetaPredict for classification analysis,
MetaClust for sparse clustering analysis,
MetaPCA for principal component analysis.

In this tutorial, 
we will go through installation and usage of MetaOmics step by step using real data examples.
The MetaOmics suite software is publicly available at \url{https://github.com/metaOmics/metaOmics}.
The tutorial itself can be found at \url{https://github.com/metaOmics/tutorial/blob/master/metaOmics_turtorial.pdf}.
Each MetaOmics module will be introduced in later sections and their R packages are also available on GitHub \url{https://github.com/metaOmics}.


 
\section{Preliminaries}
\subsection{Citing MetaOmics}
MetaOmics implements many meta-analytic methodology by their authors. 
Please cite appropriate papers when you use result from MeteOmics suit,
by which the authors will receive professional credit for their work.

\begin{itemize}
\item MetaOmics suit itself can be cited as:
\item MetaQC: \bibentry{kang2012metaqc}.
\item MetaDE: 
\begin{itemize}
\item \bibentry{fisher1925statistical}.
\item \bibentry{li2011adaptively}.
\item \bibentry{choi2003combining}.
\item and many more
\end{itemize}
\item MetaPath: 
\begin{itemize}
\item \bibentry{shen2010meta}.
\item \bibentry{fang2016cpi}.
\end{itemize}
\item MetaClust: \bibentry{huo2016meta}.
\item MetaPCA: \bibentry{kim2017metaPCA}.
\item MetaKTSP: \bibentry{Kim2016}.
\item MetaDCN: \bibentry{zhu2016metadcn}.
\end{itemize}



\subsection{Installation}
The full instruction of how to install, start are available at \url{https://github.com/metaOmic/metaOmics}.
\subsubsection{Requirement}
\begin{itemize}
\item R $>=$ 3.3.1
\item Shiny $>=$ 0.13.2
\end{itemize}

\subsubsection{How to start the app}
\begin{itemize}
\item First, clone the project
\item git clone https://github.com/metaOmic/metaOmics
\item in R (suppose the application directory is metaOmics),

$>$ install.packages($'$shiny$'$)

$>$ shiny::runApp(`metaOmics', port=9987, launch.browser=T)
\end{itemize}



\subsection{Question and bug report}
{
\color{red}
Who should be responsible for maintaining the software?
}

 
 
 
 
\section{Prepare data}
\label{sec:dataPrepare}
\subsection{Raw data}

Data should be prepared as the example in Figure~\ref{fig:dataMicroarray}.
First column should be feature ID (e.g. gene symbol) and the rest of the columns are samples.
The first row is sample ID.
Valid data type includes continuous, count.

\begin{figure}[H]
\begin{center}
\includegraphics[scale=0.5]{./figure/dataMicroarray}
\caption{A example data format}
\label{fig:dataMicroarray}
\end{center}
\end{figure}

\subsection{Clinical data}

Clinical data should be prepared as the example in Figure~\ref{fig:clinical}.
First column should be sample ID and each row represents a sample.
The rest of the columns are clinical information.

\begin{figure}[H]
\begin{center}
\includegraphics[scale=0.5]{./figure/clinicalData}
\caption{A example clinical data format}
\label{fig:clinical}
\end{center}
\end{figure}

\section{MetaOmics suit GUI}
\subsection{Setting}

After starting metaOmics, 
the first page is the metaOmics setting page in Figure~\ref{fig:GUIsetting}.  
There are 4 tabs on top of the page: Setting, Preprocessing, Saved Data and Toolsets.
Below the 4 tabs, 
the first header is the session information.
{
\color{red}
Why do we need session information?
}
The second header is Directory for Saving Output Files.
By clicking $\ldots$,
user can set default working directory, in which all the meta-analysis results will be saved.
User can view their current working directory on the top right corner.
The third header is Toolsets,
here users can view if individual packages are installed.
If the packages are installed, there is a checked installed status.
Otherwise, users can install individual package by clicking install blue button.

\begin{figure}[H]
\begin{center}
\includegraphics[scale=0.35]{./figure/GUIsetting}
\caption{GUI setting page}
\label{fig:GUIsetting}
\end{center}
\end{figure}

\subsection{Preprocessing}
If users go to the Preprocessing page as Figure~\ref{fig:GUIpreprocessing},
\begin{figure}[!htbp]
\begin{center}
\includegraphics[scale=0.35]{./figure/GUIpreprocessing}
\caption{GUI Preprocessing page}
\label{fig:GUIpreprocessing}
\end{center}
\end{figure}
they are able to uploaded genomic data via the tab ``Choosing/Upload Expression Data".
The data should be prepared according to Section~\ref{sec:dataPrepare}.
Users may optionally upload Clinical Data, depending on purpose.
{
\color{red}
Check data requirement table?.
}
After uploading is complete,
users can preview their data on the right hand side of the page as Figure~\ref{fig:GUIpreview}.
There are several Expression Data Parsing Option available on the left panel.
\begin{figure}[H]
\begin{center}
\includegraphics[scale=0.35]{./figure/GUIpreview}
\caption{GUI Preprocessing page}
\label{fig:GUIpreview}
\end{center}
\end{figure}
The MetaOmics suit also provide handlers for feature annotation, missing value imputation and multiple probe same genes.
Then users could specify type of data and study name.
Then click ``save single study" button, single study will be saved.

\subsection{Saved Data}
After uploading multiple studies w/o clinical data,
Users can turn to the Saved Data tab.
Users should select multiple datasets as Figure~\ref{fig:GUImerge}.
\begin{figure}[H]
\begin{center}
\includegraphics[scale=0.35]{./figure/GUImerge}
\caption{GUI Preprocessing page}
\label{fig:GUImerge}
\end{center}
\end{figure}
Users can select filtering criteria, enter merged study name and click on the Merge from Selected Datasets.
A merged dataset will appear on the left ``List of saved data" panel.
The last thing users need to do before using meta-analytic toolsets is select merged data and click on 
``Make merged Active Dataset" - A big green button.
Then the merged data becomes active study and shows up on the top right corner.

\section{Toolsets}


\subsection{MetaQC}

\subsection{MetaDE}

\subsection{MetaPath}

\subsection{MetaClust}
By clicking toolsets and then metaClust,
users are directed to metaClust home page as Figure~\ref{fig:metaClustHome}.
\begin{figure}[H]
\begin{center}
\includegraphics[scale=0.35]{./figure/metaClust/metaClustHome}
\caption{GUI Preprocessing page}
\label{fig:metaClustHome}
\end{center}
\end{figure}
On the top left panel users can see data summary Table.
Below there are 4 tabs.
\subsubsection{About}
About tab includes basic introduction of metaClust.
Starting with multiple studies, 
we could run MetaSparseKmeans with pre-specified number of clusters (K) and gene selection tuning parameter (Wbounds).
If you are not sure about what are good K and Wbounds, please try Tune K and Tune Wbounds panel.

\subsubsection{Tune K}
If the users are not sure what is number of clusters,
they can start to use the Tune K panel as in Figure~\ref{fig:metaClusttuneK}.
\begin{figure}[H]
\begin{center}
\includegraphics[scale=0.35]{./figure/metaClust/tuneK}
\caption{GUI Preprocessing page}
\label{fig:metaClusttuneK}
\end{center}
\end{figure}
Users will use gap statistics to get optimal K for each individual study.
Users need to specify maximum number of K, which the algorithm will search number of studies from 1 to K.
Top percentage p\% by larger variance means that we will use top p\% larger variance genes to perform gap statistics.
Number of permutation is number of bootstrap samples for gap statistics.
After selecting studies to be tuned and clicking button ``Tune K",
we will obtain gap statistics plat as in Figure~\ref{fig:metaClusttuneK}.
A good K is selected such that the $\mbox{Gap}_k$ is maximized or stablized.
From the figure, K=3 is prefered.


\subsubsection{Tune Wbounds}
Wbounds directly control number of features selected by metaClust.
If the users are not sure what is a good Wbound,
they can start to use the Tune Wbounds panel as in Figure~\ref{fig:metaClusttuneW}.
\begin{figure}[H]
\begin{center}
\includegraphics[scale=0.35]{./figure/metaClust/tuneW}
\caption{GUI Preprocessing page}
\label{fig:metaClusttuneW}
\end{center}
\end{figure}
Again,
gap statistics will be used for tuning Wbounds.
Users will specify number of clusters for tuning Wbounds, which could be obtained from the previous step.
Iterations is the same thing as number of bootstrap samples for gap statistics.
Users also need to specify the searching space of Wbounds by minimum of Wbounds, maximum of Wbounds and Step of Wbounds.
After all these steps are set,
user can click on ``Tune Wbounds" button.
The results will be shown in Figure~\ref{fig:metaClusttuneW}.
Wbound=12 is preferred since the corresponding gap statistics is maximized.

\subsubsection{Run Meta Sparse K-Means}
Under Run Meta Sparse K-Means panel,
user can specify number of clusters, Wbounds and run meta sparse K means, 
as in Figure~\ref{fig:mskmRes}.
\begin{figure}[H]
\begin{center}
\includegraphics[scale=0.35]{./figure/metaClust/mskmRes}
\caption{GUI Preprocessing page}
\label{fig:mskmRes}
\end{center}
\end{figure}
There are three clustering matching methods: Exhaustive, linear, MCMC.
Exhaustive is suggested if the data is not large.
Linear will perform smart search and get solution much faster than Exhaustive, 
but it may yield less accuracy.
MCMC might by very time consuming.
Adjust sample size checkbox allows users to adjust sample size effect.
After number of clusters and Wbounds are specified,
users can click on Run meta sparse K means and obtain results as Figure~\ref{fig:mskmRes}.

\subsection{MetaPCA}

\subsection{MetaKTSP}

\subsection{MetaDCN}

\subsection{MetaLA}

%%  for citation purpose
\bibliographystyle{apalike}
\bibliography{reference}
 
\end{document}
\documentclass{article}

\usepackage[utf8]{inputenc}
\usepackage{hyperref}
\usepackage{natbib}
\usepackage{bibentry}
\usepackage{color}
\usepackage{graphicx}

\nobibliography*

\title{A tutorial for metaOmic}
\author{}
\date{ }
 
 
\begin{document}
 
\maketitle
 
\tableofcontents
 
\section{Introduction}
 
MetaOmics software suite is an interactive software with graphical user interface (GUI) for genomic meta-analysis implemented using R shiny.
Many state of art meta-analysis tools are available in this software,
including MetaProcess for omics data preprocessing, 
MetaQC for quality control, 
MetaDE for differential expression analysis,
MetaPath for pathway enrichment analysis,
MetaNetwork for differential co-expression network analysis,
MetaPredict for classification analysis,
MetaClust for sparse clustering analysis,
MetaPCA for principal component analysis.

In this tutorial, 
we will go through installation and usage of MetaOmics step by step using real data examples.
The MetaOmics suite software is publicly available at \url{https://github.com/metaOmics/metaOmics}.
The tutorial itself can be found at \url{https://github.com/metaOmics/tutorial/blob/master/metaOmics_turtorial.pdf}.
Each MetaOmics module will be introduced in later sections and their R packages are also available on GitHub \url{https://github.com/metaOmics}.


 
\section{Preliminaries}
\subsection{Citing MetaOmics}
MetaOmics implements many meta-analytic methodology by their authors. 
Please cite appropriate papers when you use result from MeteOmics suit,
by which the authors will receive professional credit for their work.

\begin{itemize}
\item MetaOmics suit itself can be cited as:
\item MetaQC: \bibentry{kang2012metaqc}.
\item MetaDE: 
\begin{itemize}
\item \bibentry{fisher1925statistical}.
\item \bibentry{li2011adaptively}.
\item \bibentry{choi2003combining}.
\item and many more
\end{itemize}
\item MetaPath: 
\begin{itemize}
\item \bibentry{shen2010meta}.
\item \bibentry{fang2016cpi}.
\end{itemize}
\item MetaClust: \bibentry{huo2016meta}.
\item MetaPCA: \bibentry{kim2017metaPCA}.
\item MetaKTSP: \bibentry{Kim2016}.
\item MetaDCN: \bibentry{zhu2016metadcn}.
\end{itemize}



\subsection{Installation}
The full instruction of how to install, start are available at \url{https://github.com/metaOmic/metaOmics}.
\subsubsection{Requirement}
\begin{itemize}
\item R $>=$ 3.3.1
\item Shiny $>=$ 0.13.2
\end{itemize}

\subsubsection{How to start the app}
\begin{itemize}
\item First, clone the project
\item git clone https://github.com/metaOmic/metaOmics
\item in R (suppose the application directory is metaOmics),

$>$ install.packages($'$shiny$'$)

$>$ shiny::runApp(`metaOmics', port=9987, launch.browser=T)
\end{itemize}



\subsection{Question and bug report}
{
\color{red}
Who should be responsible for maintaining the software?
}

 
 
 
 
\section{Prepare data}
\subsection{Raw data}

Data should be prepared as the example in Figure~\ref{fig:dataMicroarray}.
First column should be feature ID (e.g. gene symbol) and the rest of the columns are samples.
The first row is sample ID.
Valid data type includes continuous, count.

\begin{figure}[htbp]
\begin{center}
\includegraphics[scale=0.5]{./figure/dataMicroarray}
\caption{A example data format}
\label{fig:dataMicroarray}
\end{center}
\end{figure}

\subsection{Clinical data}

Clinical data should be prepared as the example in Figure~\ref{fig:clinical}.
First column should be sample ID and each row represents a sample.
The rest of the columns are clinical information.

\begin{figure}[htbp]
\begin{center}
\includegraphics[scale=0.5]{./figure/clinicalData}
\caption{A example clinical data format}
\label{fig:clinical}
\end{center}
\end{figure}



\section{Preprocessing}

\section{MetaQC}

\section{MetaDE}

\section{MetaPath}

\section{MetaClust}

\section{MetaPCA}

\section{MetaKTSP}

\section{MetaDCN}

\section{MetaLA}

%%  for citation purpose
\bibliographystyle{apalike}
\bibliography{reference}
 
\end{document}
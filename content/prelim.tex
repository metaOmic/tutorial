\section{Preliminaries}
\subsection{Citing MetaOmics}
MetaOmics software suite implements many meta-analytic methodologies from different authors. 
Please cite appropriate papers if you use MeteOmics,
by which the authors will receive professional credits for their work.

\begin{itemize}

\item MetaOmics software suite  itself can be cited as: 

\begin{itemize}
\item Ma et al. MetaOmics: Comprehensive Analysis Pipeline and Browser-based Software Suite for Transcriptomic Meta-Analysis.
\end{itemize}

\item MetaQC: 
\begin{itemize}
\item \bibentry{kang2012metaqc}.
\end{itemize}

\item MetaDE: 
\begin{itemize}
\item \bibentry{fisher1925statistical}.
\item \bibentry{li2011adaptively}.
\item \bibentry{huo2017p}.
\item \bibentry{choi2003combining}.
\item \bibentry{song2014hypothesis}.
\item \bibentry{lu2009biomarker}.
\end{itemize}

\item MetaPath: 
\begin{itemize}
\item \bibentry{shen2010meta}.
\item \bibentry{fang2016cpi}.
\end{itemize}

\item MetaNetwork: 
\begin{itemize}
\item \bibentry{zhu2016metadcn}.
\end{itemize}

\item MetaPredict: 
\begin{itemize}
\item \bibentry{Kim2016}.
\end{itemize}

\item MetaClust: 
\begin{itemize}
\item \bibentry{huo2016meta}.
\end{itemize}

\item MetaPCA: 

\begin{itemize}
\item \bibentry{kim2017metaPCA}.
\end{itemize}

\end{itemize}



\subsection{How to start MetaOmics}

The full instruction of how to install, start MetaOmics software suite is also available at \url{https://github.com/metaOmics/metaOmics}.


\subsubsection{Requirement}
\begin{itemize}
\item R $>=$ 3.3.1
\item Shiny $>=$ 0.13.2
\end{itemize}



{\bf Note:}
\begin{itemize}
\item We recommend users to use R 3.3 to implement our tool. If you are using R 3.4 (newly released), you may encounter errors in installing dependencies of the modules. You can manually install the dependencies by running the following commands in R:

\textit{install.packages(c(\textquotesingle GSA\textquotesingle, \textquotesingle combinat\textquotesingle, \textquotesingle   samr\textquotesingle   , \textquotesingle   survival\textquotesingle   , \textquotesingle   cluster\textquotesingle   , \textquotesingle   gplots\textquotesingle   , 
  \textquotesingle   ggplot2\textquotesingle   , \textquotesingle   irr\textquotesingle   , \textquotesingle   shape\textquotesingle   , \textquotesingle   snow\textquotesingle   , \textquotesingle   snowfall\textquotesingle   , \textquotesingle   igraph\textquotesingle   , \textquotesingle   doMC\textquotesingle   , \textquotesingle   PMA\textquotesingle   ))
  }

\textit{source(\textquotesingle   https://bioconductor.org/biocLite.R\textquotesingle   )  }

\textit{biocLite(c(\textquotesingle   multtest\textquotesingle   , \textquotesingle   Biobase\textquotesingle   , \textquotesingle   edgeR\textquotesingle   , \textquotesingle   DESeq2\textquotesingle   , \textquotesingle   impute\textquotesingle   , 
  \textquotesingle   limma\textquotesingle   , \textquotesingle   AnnotationDbi\textquotesingle   , \textquotesingle   ConsensusClusterPlus\textquotesingle   , \textquotesingle   genefilter\textquotesingle   , \textquotesingle   GSEABase\textquotesingle   , \textquotesingle   Rgraphviz\textquotesingle   ))
  }

\item For Windows, users need to run the following command in R to install the package \textquotesingle doMC\textquotesingle:

\textit{install.packages(\textquotesingle doMC\textquotesingle, repos=\textquotesingle http://R-Forge.R-project.org\textquotesingle)}

\end{itemize}

 

\subsubsection{How to install the metaOmics software}
\begin{itemize}
\item At MetaOmics home page at \url{https://github.com/metaOmics/metaOmics}, clone the project by
clicking on ``Clone or download" and extract to a working directory, 
or type in the following in command line:

\textit{git clone} \url{https://github.com/metaOmic/metaOmics}
\end{itemize}

\subsubsection{How to start the metaOmics software}
\begin{itemize}
\item In R (suppose the application directory is metaOmics),

\textit{install.packages(\textquotesingle shiny\textquotesingle)}

\textit{shiny::runApp(\textquotesingle metaOmics\textquotesingle, port=9987, launch.browser=T)}
\end{itemize}

\subsection{MetaOmics setting page}
\label{sec:setting}
After starting MetaOmics, 
the first page is the MetaOmics setting page as shown in Figure~\ref{fig:GUIsetting}.  
There are 4 tabs on top of the page (at position {\color{red} (1)}): Setting, Preprocessing, Saved Data and Toolsets.
The welcome page is below the 4 tabs, where contains authors' information.
Further below, the first header is the session information.
The second header is Directory for Saving Output Files (at position {\color{red} (2)}).
By clicking ``$\ldots$",
user can set default working directory, in which all the meta-analysis results will be saved.
Users can view their current working directory on the top right corner (at position {\color{red} (3)}).
The third header is Toolsets (at position {\color{red} (4)}),
where user can click to install desired modules if the ``status" shows ``not installed".
If the packages are installed, there is a checked installed status.
Otherwise, users can install individual package by clicking install blue button.
The installation progress may take a few minutes for each module.
There will be a notification icon at the bottom right corner after the installation. 
After the modules are installed to R, restart the MetaOmics software suite so that the Shiny application interface is updated with installed modules.
Position {\color{red} (5)} shows the current active dataset, which will be introduced in Section~\ref{sec:procedure}~\ref{sec:active}. 
 
\begin{figure}[H]
\begin{center}
\includegraphics[scale=1]{./figure/preprocessing/GUIsetting}
\caption{MetaOmics software suite GUI setting page}
\label{fig:GUIsetting}
\end{center}
\end{figure}


\subsection{Question and bug report}

If you encounter errors or bugs, please report to maintainer Tianzhou Ma \url{tim28@pitt.edu}.


 

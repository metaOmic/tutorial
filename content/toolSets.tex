\section{Toolsets}

After the MetaPreprocess,
all the preparatory steps are done. 
It's time to apply seven metaOmics modules by clicking on the ``Toolsets" tab and select the tool for your research question.
In the next few subsections, 
we will introduce in details how to run these modules.
For each module, a summary table to studies and sample sizes is shown on the top left corner. 
There is an ``about" drop-down menu which contains brief introduction and tutorial associated with the module.
The ``options" drop-down menu contains common options users can select or tune in the analysis.
The ``advanced options" section are more technical which we generally do not recommend users to change unless they are familiar with the methods.
After applying these metaOmics modules, all result files will be automatically saved in the working directory which is specified in Section~\ref{sec:setting}.
For computationally demanding methods, the procedure may take minutes or up to hours depending on size of datasets.
Users can keep track of the progress by checking the R console.







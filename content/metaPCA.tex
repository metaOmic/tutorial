\subsection{MetaPCA}
Dimension reduction is a popular data-mining approach for transcriptomic analysis.
MetaPCA aims to combine multiple omics datasets of identical or similar biological hypothesis and perform simultaneous dimensional reduction in all studies.
The results show improved accuracy, robustness, and better interpretation among all studies.
By clicking the ``Toolsets" tab and then choosing MetaPCA,
users are directed to the MetaPCA homepage, as shown in Figure~\ref{fig:metaPCAHome}.
The R package for the MetaPCA module can be found at \url{https://github.com/metaOmics/metaPCA}.

\begin{figure}[H]
\begin{center}
\includegraphics[scale=0.4]{./figure/metaPCA/metaPCAHome.pdf}
\caption{MetaPCA settings}
\label{fig:metaPCAHome}
\end{center}
\end{figure}

\subsubsection{Procedure}

The procedure on describing how to use metaPCA is described below.
A complete list of options is available in Section~\ref{sec:completeList_MetaPCA}.

\begin{steps}

\item \textbf{Specify parameters} 

There are very few parameters that need to be specified for MetaPCA, as in Figure~\ref{fig:metaPCAHome}.
Advanced options are not suggested to be changed unless the users are familiar with the algorithm.
There are two methods for MetaPCA (at position {\color{red} (1)}). 
SSC represents MetaPCA via sum of squared cosine (SSC) maximization.
SV represents MetaPCA via sum of variance decomposition (SV).
Details of SSC and SV can be found in MetaPCA manuscript \citep{kim2017meta}.
SSC has better performance and is suggested.
The dimension of meta-eigenvector matrix option {\color{red} (2)} allows users to specify the dimension of the output meta-eigenvector matrix.
The checkbox of ``dimension determined by variance quantile" is suggested to be checked {\color{red} (3)}.
When checked, the dimension size of each study's eigenvector matrix (SSC) is determined  by the pre-defined level of variance quantile 80\%.
If the checkbox of ``sparsity encouraged" is checked (at position {\color{red} (4)}), users can perform MetaPCA.
After clicking on the ``search for optimal tuning parameter" button, the optimum tuning parameter will be returned to the box ``tuning parameter for sparsity," 
which may be time consuming.

\item \textbf{Perform MetaPCA} 

By clicking the ``Run Meta PCA" button, the MetaPCA module will be performed.


\end{steps}


\subsubsection{Results}
The input dataset is the same as the input for MetaDE module.
%After performing merging of the three datasets and filter 50\% genes by mean and 50\% by variance, 1283 genes remained.
Detailed descriptions of these studies can be found in Table~\ref{tab:realDataLeukemia}. 

The result of MetaPCA is shown in Figure~\ref{fig:metaPCAresult},
which shows nice separations between three groups.
These figures and eigenvectors are saved to the MetaPCA folder.

\begin{figure}[H]
\begin{center}
\includegraphics[scale=0.3]{./figure/metaPCA/metaPCA}
\caption{MetaPCA result.
The x-axis (horizontal) is the first principal component, 
and the y-axis (vertical) is the second principal component.
Each dot represents a sample in a study with the sample label marked to the top right of the figure.
}
\label{fig:metaPCAresult}
\end{center}
\end{figure}

